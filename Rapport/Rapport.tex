\documentclass{journalstyle}




\title{Tests for an Increasing Trend in the Intensity of a Poisson Process: A Power Study}
\author{Maxime Baba, Mathile Ferrera, Félix Foucher de Brandois}
\newcommand{\footertext}{Formation ModIA - INSA, 5$^{\text{th}}$ year \\ 2024-2025}

\begin{document}


\maketitle

%\tableofcontents
%\listoffigures



\section{Introduction}

A Nonhomogeneous Poisson Process (NHPP) is a stochastic process often used to model phenomena where the rate of occurrence of events changes over time.
The rate, or intensity function $\lambda(t)$, represents the expected number of events per unit time at a given time $t$.
Understanding and analyzing the behavior of this rate is crucial in diverse fields such as reliability engineering, healthcare, and environmental studies, as it helps identify patterns, predict future events, and optimize interventions.
Detecting increasing trends in the rate can be particularly important, for example, in monitoring system deterioration or identifying escalating risks in processes.
Over the years, a variety of statistical methods have been proposed to identify increasing trends in NHPPs.
This study aims to extend the work of Bain, Engelhardt, and Wright \cite{BainEngelhardtWright} by providing a detailed theoretical and practical exploration of the Laplace test and Boswell’s likelihood ratio test.
The paper is structured in two main parts: a presentation of the selected tests, including their theoretical basis and simulated performance comparisons, and a practical application to real-world data.


\section{Presentation of the Tests}

Explication du principe de test : $H_0$ contre $H_1$.

\subsection{Laplace Test}

Présentation du principe général du test de Laplace.

\subsubsection{Theoretical Basis}

Let $(N(t))_{t \geq 0}$ be a Poisson process with intensity function $\lambda(t)$.
We observe $N(t)$ in the interval $[0, T^*]$.
Let $0 < T_1 < T_2 < \ldots < T_n < T^*$ be the ordered observation times.

\noindent\underline{Hypotheses} \\
$H_0$ : $\lambda(\cdot)$ is constant versus
$H_1$ : $\lambda(\cdot)$ is increasing.

\noindent\underline{Test Statistic} \\
Let $F = \frac{1}{T^*} \sum_{i=1}^n T_i$. \\
Under $H_0$, $(T_1, \ldots, T_n) | \{N_{T^*} = n\} \overset{(d)}{=} (U_1, \ldots, U_n)$, where $U_1, \ldots, U_n \underset{i.i.d.}{\sim} \mathcal{U}([0, T^*])$. \\
Therefore, $(\frac{T_1}{T^*}, \ldots, \frac{T_n}{T^*}) | \{N_{T^*} = n\} \overset{(d)}{=} (V_1, \ldots, V_n)$, where $V_1, \ldots, V_n \underset{i.i.d.}{\sim} \mathcal{U}([0, 1])$. \\
According to the central limit theorem, we have :
\[
\sqrt{n} \left( \frac{F}{T^*} - \frac{1}{2} \right) \xrightarrow[n \to \infty]{(d)} \mathcal{N}(0, \frac{1}{12})
\]



\subsubsection{Implementation}


\subsection{Boswell's Likelihood Ratio Test}

\subsubsection{Theoretical Basis}

\subsubsection{Implementation}


\section{Application to Real-World Data}

\subsection{Application Scenario}

\subsection{Power comparison}

\subsection{Analysis of the Results}




\section{Conclusion}
Ceci est une citation \cite{smith2020example}.

\newpage sqd


\printbibliography


\end{document}