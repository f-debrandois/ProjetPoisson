\documentclass[12pt,a4paper]{article}
\usepackage[utf8]{inputenc} 
\usepackage[T1]{fontenc}		       
\usepackage{lmodern}			       
\usepackage{babel}
\usepackage{amsmath}
\usepackage{amsfonts}
\usepackage{amssymb}
\usepackage{graphicx}
\usepackage{xcolor}
\usepackage{mathtools}
\usepackage{fancyhdr}
\usepackage{enumitem}
\usepackage{tcolorbox}
\usepackage{stmaryrd}
\usepackage{dsfont}
\usepackage{pgf, tikz}
\usetikzlibrary{shapes.misc}
\usepackage[linesnumbered,ruled,vlined]{algorithm2e}
\usepackage[text={15cm,24.5cm},centering]{geometry}


% Définir la fonction pour créer une boîte de code
\newcommand{\code}[1]{%
    \begin{tcolorbox}[colback=black!10!white,colframe=black]
        #1
    \end{tcolorbox}
}


\begin{document}

\begin{figure}[t]
    \centering
    \includegraphics[width=5cm]{src/inp_n7.png}
    \hfill
    \includegraphics[width=3.8cm]{src/insa_toulouse.png}
\end{figure}


\title{\vspace{4cm} \textbf{Rapport BE : \\ 
Testing for homogeneity of a Poisson process}}

\author{by Maxime Baba, \\
        Mathilde Fererra, \\
        and Felix Foucher de Brandois}
        
\date{\vfill Formation ModIA - INSA, 5$^{th}$ year \\
2024-2025}

\maketitle

\newpage
\tableofcontents
\listoffigures

\newpage


\section{Introduction}


\section{Brouillon}


Soit $(N_t)_t$ un processus de Poisson inhomogène de taux $\lambda(t)$, $t \in \mathbb{R}$. \\

On observe ce processus sur un intervalle $[0, T^*]$, $T^* > 0$. \\
Soit $n = N_{T^*}$ le nombre d'événements observés. \\
Soit $0 < T_1 < T_2 < \ldots < T_n < T^*$ les instants des événements observés. \\

\underline{Test de Laplace} : \\
On teste l'hypothèse $H_0 : \lambda(t) = \lambda_0$ pour tout $t \in [0, T^*]$. \\
contre \\
$H_1 : \lambda$ croissante sur $[0, T^*]$. \\

Stat de test :
$$
L = \sum_{i=1}^n \frac{T_i}{T^*}
$$

\textbf{Conditional distribution} : \\
Let $N$ be a homogeneous Poisson process with rate $\lambda > 0$
and fix $t > 0$. Let $n \in \mathbb{N}^*$.
Given that $N_t = n$, the $n$ first arrival times $(T_1, \ldots, T_n)$ have the same distribution as the order statistic corresponding to $n$ independent random variables uniformly distributed on the interval $[0, t]$,
that is : \\
$$
(T_1, T_2, \ldots, T_n)|{N_t = n} \overset{d}{=} (U_{(1)}, U_{(2)}, \ldots, U_{(n)}) \quad \text{where} \quad U_1, \ldots, U_n \text{ i.i.d} \sim U([0, t])
$$ \\

Sous $H_0$, les variables aléatoires $\frac{T_i}{T^*}$ sont i.i.d et ont la même distribution que la statistique d'ordre correspondant à $n$ variables aléatoires uniformément distribuées sur $[0, 1]$. \\
Donc sous $H_0$, $L \xrightarrow[n \to \infty]{\mathcal{L}} \mathcal{N}(0, 1)$ (théorème de la limite centrale). \\

H0 : $\beta = 0$ contre H1 : $\beta > 0$ \\
pour $\lambda(t) = \alpha e^{\beta t}$, 



\end{document}