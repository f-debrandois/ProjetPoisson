\documentclass{journalstyle}




\title{Tests for an Increasing Trend in the Intensity of a Poisson Process: A Power Study}
\author{Maxime Baba, Mathile Ferrera, Félix Foucher de Brandois}
\newcommand{\footertext}{Formation ModIA - INSA, 5$^{\text{th}}$ year \\ 2024-2025}

\begin{document}


\maketitle

%\tableofcontents
%\listoffigures



%\section*{Introduction}

A Nonhomogeneous Poisson Process (NHPP) is a stochastic process often used to model phenomena where the rate of occurrence of events changes over time.
The rate, or intensity function $\lambda(t)$, represents the expected number of events per unit time at a given time $t$.
Understanding and analyzing the behavior of this rate is crucial in diverse fields such as reliability engineering, healthcare, and environmental studies, as it helps identify patterns, predict future events, and optimize interventions.
Detecting increasing trends in the rate can be particularly important, for example, in monitoring system deterioration or identifying escalating risks in processes.
Over the years, a variety of statistical methods have been proposed to identify increasing trends in NHPPs.
This study aims to extend the work of Bain, Engelhardt, and Wright \cite{BainEngelhardtWright} by providing a detailed theoretical and practical exploration of the Laplace test and Boswell’s likelihood ratio test.
The paper is structured in two main parts: a presentation of the selected tests, including their theoretical basis and simulated performance comparisons, and a practical application to real-world data.

\textit{Changes to be made:}
\begin{itemize}
    \item Ordre de présentation de l'article
    \item Source sur les exemples (healthcare, ...)
\end{itemize}

\section{Theoretical Framework for Trend Detection Tests}

A statistical test evaluates two competing hypotheses: the null hypothesis $H_0$ (representing the status quo) and the alternative hypothesis $H_1$ (indicating a deviation from $H_0$).
Based on the sample data, a test statistic is computed and compared to a critical threshold.
If the test statistic exceeds this threshold, $H_0$ is rejected in favor of $H_1$. \\
In our case, we test whether the intensity function $\lambda(t)$ of a Poisson process is constant ($H_0$) or increasing ($H_1$).

\subsection{Laplace Test}

\subsubsection{Theoretical Basis}

Let $(N(t))_{t \geq 0}$ be a Poisson process with intensity function $\lambda(t)$.
We observe $N(t)$ in the interval $[0, T^*]$.
Let $0 < T_1 < T_2 < \ldots < T_n < T^*$ be the ordered observation times.

\noindent\textbf{Test Statistic} \\
Under $H_0$, the arrival times $T_1, \ldots, T_n$ (conditioned on $N_{T^*} = n$) behave like order statistics from a uniform distribution.
Specifically: \\
$(T_1, \ldots, T_n) | \{N_{T^*} = n\} \overset{(d)}{=} (U_1, \ldots, U_n)$, where $U_1, \ldots, U_n \underset{i.i.d.}{\sim} \mathcal{U}([0, T^*])$. \\
Therefore, $(\frac{T_1}{T^*}, \ldots, \frac{T_n}{T^*}) | \{N_{T^*} = n\} \overset{(d)}{=} (V_1, \ldots, V_n)$, where $V_1, \ldots, V_n \underset{i.i.d.}{\sim} \mathcal{U}([0, 1])$. \\
Define the Laplace test statistic as: 
\begin{equation}
    F = \frac{1}{T^*} \sum_{i=1}^n T_i.
    \label{eq:laplace_test_statistic}
\end{equation}

By the Central Limit Theorem, under $H_0$,  can be standardized as: 
$Z = \frac{F - \mathbb{E}[F]}{\sqrt{\text{Var}(F)}} = \frac{F - \frac{n}{2}}{\sqrt{\frac{n}{12}}} \sim \mathcal{N}(0, 1)$
for large $n$.

\noindent\textbf{Decision Rule} \\
If the intensity $\lambda(t)$ is increasing, the arrival times $T_i$ are expected to cluster towards the end of the interval $[0, T^*]$, making $F$ larger.
Therefore, we reject $H_0$ if $F \geq s_{\alpha}$, where $s_{\alpha}$ is the critical threshold determined as: $s_{\alpha} = \frac{n}{2} + z_{1 - \alpha} \sqrt{\frac{n}{12}}$. \\
with $z_{1 - \alpha}$ being the $(1 - \alpha)$-th quantile of the standard normal distribution.

\noindent\textbf{Type I Error} \\
$e_1 = \mathbb{P}_{H_0}(F \geq s_{\alpha}) = \mathbb{P}(Z \geq z_{1 - \alpha}) = \alpha$.

\noindent\textbf{P-value} \\
The p-value measures the probability of obtaining a test statistic at least as extreme as the observed $F_{\text{obs}}$, under $H_0$.
It is given by: \\
$\hat{\alpha} = \mathbb{P}_{H_0}(F \geq F_{\text{obs}}) = 1 - \Phi(\frac{F_{\text{obs}} - \frac{n}{2}}{\sqrt{\frac{n}{12}}})$.

\noindent\textbf{Power} \\
The power of the test is the probability of rejecting $H_0$ when $H_1$ is true. \\
$\pi(\lambda) = \mathbb{P}_{H_1}(F \geq s_{\alpha})$.


\subsection{Boswell's Likelihood Ratio Test}

\subsubsection{Theoretical Basis}


\section{Numerical Simulations}

\subsection{Simulation Methodology}

Generating simulated data : \\
Multiple rates functions...

\subsection{Power Analysis}
Calculation of the power of the Laplace and Boswell tests under different conditions (e.g., exponential, Weibull, and step-function trends). \\

Visualization of power curves to compare test effectiveness. \\

Discussion of factors influencing the power of each test (sample size, intensity function shape, etc.). \\

Insights into when to prefer one test over the other.

\section{Application to Real-World Data}

\subsection{Description of the Dataset}
Danish Fire Insurance Claims. \\
Overview of the dataset containing large fire insurance claims in Denmark from 1980 to 1990. \\
Explanation of how the event times are extracted and processed.

\subsection{Applying the Tests}
Presentation of test statistics, p-values, and decisions. \\

\subsection{Results and Discussion}
Analysis of Findings. \\
Interpretation of the results and their implications for the intensity of fire insurance claims. \\
Discussion of whether an increasing trend is detected and its significance. \\
Comparison with Simulated Results. \\
Reflecting on how real-world results compare to the numerical simulations.

\newpage qsd



\section{Conclusion}
Ceci est une citation \cite{smith2020example}.



\printbibliography


\end{document}